\chapter{Implementation: LSP server architecture} \label{lspimplementation}

The following chapters will discuss how CCDetect-LSP is implemented, and
especially focuses on our novel algorithm based on dynamic extended suffix arrays. The
current chapter will look at the LSP integration and how source code is initially indexed,
while the following two chapters will first discuss how clones are initially detected and
then how this analysis is incrementally updated as source code changes.

CCDetect-LSP is integrated into IDEs via LSP. The tool starts up as an LSP server, which
an IDE client can connect to and send/receive messages from. The goal of the LSP
client-server interaction is to give users of the tool an overview of all clones as they
appear in source code, and allow them to navigate between matching clones. CCDetect-LSP is
implemented in Java with the LSP4J library which provides an abstraction layer on top of
the protocol which is easier to work with programmatically.

\begin{figure}[ht!]
	\includegraphics[width=\textwidth]{figures/vscodecodeclone.png}
	\caption{Code clones displayed in the VSCode IDE}
	\label{fig:vscodeclones}
\end{figure}

\begin{figure}[t]
	\includegraphics[width=\textwidth]{figures/architecture.drawio.pdf}
	\caption{Tool architecture}
	\label{fig:architecture}
\end{figure}


Figure \ref{fig:vscodeclones} shows how clones are displayed in the VSCode IDE. The image
shows a section of a file where a Java switch statement has been duplicated in two other
files. Code clones are displayed inline in the file as an ``Information'' diagnostic, and
when hovered over, the client shows the \verb|DiagnosticRelatedInformation| information of
the diagnostic where all the matching clones are displayed and can be clicked to navigate
to the corresponding match. For a client which may not implement the
\verb|DiagnosticRelatedInformation| message\footnote{Added to LSP in version 3.16,
released in 2020. In our experience, some IDEs do not implement this feature.}, invoking a
code-action to navigate to the matching clone is another option.

The following user-stories shows how interaction with the LSP server works.

\begin{itemize} 

    \item A programmer wants to see code clones for a file in their project, the
        programmer opens the file in their IDE and is shown diagnostics inline with the
        code wherever there are detected clones. The matching code clones are not
        necessarily in the same file.

	\item A programmer wants to see all code clones for the current project. The
	      programmer opens the IDEs diagnostic view and will see all code clones detected
	      as diagnostics there. The diagnostic will contain information like where the clone
	      exists, and where the matching clone(s) are.

    \item A programmer wants to jump to one of the matches of a code clone in their IDE.
        The programmer moves their cursor to the diagnostic and will see a list of the
        matching code clones. The programmer selects any of the code clones which will
        open the file and location of the selected code clone. Alternatively, a
        code-action can be invoked to navigate, if the client does not implement the
        \verb|DiagnosticRelatedInformation| information.


      \item A programmer wants to refactor a set of clones by applying the
          ``extract-method'' refactoring (not implemented by CCDetect-LSP). The programmer
          performs the necessary refactorings, saves the file and will see that the
          refactored clones are no longer detected when the LSP server incrementally
          updates its analysis.

      \item A programmer is editing code and has written or copy-pasted a function which
          was already present in the source code of a large code base. As soon as the
          programmer saves the file, CCDetect-LSP will perform an incremental update based
          on the change, and display that the duplicated source code is a clone. The clone
          will be displayed until it is removed or modified until it is no longer a clone.

      \item A programmer wants to switch to a project in a different programming language,
          and wants to use CCDetect-LSP to analyze clones in that project. The programmer
          edits the CCDetect-LSP configuration sent from the client to select the new
          language, which involves changing two parameters in the configuration. The
          configuration is explained in more detail in \cref{configuration}. If the
          programmer wants to add a new language which has not yet been added in the
          CCDetect-LSP source code, a grammar for the language needs to be added before
          CCDetect-LSP can analyze that language.

      \item A programmer wants to use CCDetect-LSP in a new IDE. The process of running
          CCDetect-LSP in an IDE is different for each IDE. In general, an IDE which
          implements LSP, will have some sort of setup for launching an LSP server with a
          command, or on a certain event. The setup for CCDetect-LSP will be the same as
          for any other LSP server, but if extra configuration beyond the default values
          is wanted (such as changing language), the set of configuration options need to
          be sent from the client, as explained in \cref{configuration}.

\end{itemize}


Figure \ref{fig:architecture} shows the architecture of the tool. The LSP module
communicates with the IDE and delegates the work of handling documents to the document
index. Detection of clones is delegated to the detection module. The LSP module receives a
list of clones from the detection module which is then converted to LSP diagnostics and
code-actions which is finally sent to the client.

\section{Document index}

Upon starting, the LSP server needs to index the project. This involves creating an index
and inserting all the relevant documents in the code base. A document consists of the
content of a file along with extra information such as the file's URI and some information
which is useful for the clone detection. We define the following record for our documents:

\begin{lstlisting}
    Document {
        String uri,
        String content,
        AST ast,

        // Location in fingerprint
        int start,
        int end

        // Used for incremental updates
        int[] fingerprint,
        boolean open,
        boolean changed,
    }
\end{lstlisting}

Each document in the index primarily consists of the contents, the URI and the AST of the
document in its current state. Storing the AST will be useful for the incremental
detection algorithm.

There are two things to consider when determining which files should be inserted into the
index. First, we are considering only files of a specific file type, since the tool does
not allow analysis of multiple programming languages at the same time\footnote{However,
    multiple instances of CCDetect-LSP can be run to analyze multiple languages in
the same IDE}. Therefore, the index should contain for example only \verb|*.java| files if
Java is the language to analyze. Second, all files of that file type might not be relevant
to consider in the analysis. This could for example be generated code, which generally
contains a lot of duplication, but is not practical or necessary to consider as duplicate
code, since this is not code which the programmer interacts with directly. Therefore, the
default behavior is to consider only files of the correct file type, which are checked
into version control. The tool supports adding all files in a folder, or all files checked
into version control.

When a document is first indexed by the server, the file contents is read from the disk.
However, as soon as the programmer opens this file in their IDE, the source of truth for
the files content is no longer on the disk, as the programmer is changing the file
continuously before writing to the disk. The LSP protocol defines multiple JSON-RPC
messages which the client sends to the server in order for the server to keep track of
which files are opened, and the state of the content of opened files.

Upon opening a file, the client will send a \verb|textDocument/didOpen| message to the
server, which contains the URI for the opened file. The index will at this point set the
flag \verb|open| for the relevant document and stops reading its contents from disk.
Instead, updates to the file are obtained via the \verb|textDocument/didChange| message.
This message can provide either the entire content of the file each time the file changes,
or it can provide only the changes and the location of the change. Receiving only the
changes will be useful for this algorithm when the analysis incrementally parses the
document.

\section{Displaying and interacting with clones}

When detection is finished and the list of clones is ready, the LSP module will convert
clones into diagnostics and code-actions which can be interacted with from the client. For
diagnostics, each clone object is converted into a diagnostic which has a range and some
information about the clone. The diagnostic displays an information message in the client
which also has information for each matching clone. This is achieved this by attaching
\verb|DiagnosticRelatedInformation| to a diagnostic, which is defined in LSP. When all the
code clones have been converted to diagnostics, the server sends a
\verb|textDocument/publishDiagnostics| message which sends all the diagnostics to the
client in JSON-RPC format. For code-actions, the process is similar, for each clone pair,
we create a code-action which navigates from one to the other, using the
\verb|window/showDocument| message. These code-actions are similarly sent to the client
via the \verb|textDocument/codeAction| message, but this message is only invoked when the
client specifically requests code actions at a certain location.

\section{Configuration}
\label{configuration}

CCDetect-LSP allows the LSP client to configure different parameters of CCDetect-LSP. The
configuration works by utilizing the \verb|intializationOptions| section of the
\verb|initialize| message, which is the first message sent from the client to the server.
The client can send initialization options to the CCDetect-LSP server, with the following
default values if none are given:

\begin{lstlisting}
{
    language = "java",
    fragment_query = "(method_declaration) @method",
    clone_token_threshold = 100,
    dynamic_detection = true,
    update_on_save = true,
    ignore_nodes = [],
    blind_nodes = [],
}


\end{lstlisting}

The fragment query option is a Tree-sitter query, which is explained in more detail in
\cref{initialdetection}. The rest of the configuration options will also become clear in
the following chapters.
