\chapter{Discussion}

This chapter will discuss the results of the different evaluations performed in
\cref{evaluation} and conclude with some insights into which algorithms seem to be more
efficient in each case. We will also discuss the implications of the memory usage, and
discuss the clones which are detected, and the possibility of extending the incremental
algorithm of CCDetect-LSP to more intricate types of clones.

\section{Performance}

In \cref{evaluation} we saw how the SACA detection, incremental detection and iClones
performed on multiple code bases of differing size. This section will discuss these
observations and consider which factors that contributed to the observed outcomes and what
this means for CCDetect-LSP as a clone detection tool in an IDE environment. While we have
tested iClones in our evaluation, we do not have access to the source code, which is not
open-source. Therefore, we cannot speak much to why iClones is faster/slower on certain
test instances. It is possible that iClones spends some time to build data structures for
detecting type-3 clones, which would be unfair for the evaluation as type-3 detection is
not considered. However, it seems like at minimum, most of the work is avoided when type-3
clone detection is disabled, as enabling it drastically slows down the benchmarks of
iClones.

It is clear that for the initial detection (revision 0), iClones has the best performance.
This was unexpected, as one would think that the SACA detection, which does not need to
build any significant data structures would be able to detect the clones faster than
iClones. However, we also see that the subsequent SACA detections (revisions 1-9) after
the initial detection is generally much faster than the initial iClones detection
(revision 0). Since much of the initial detection time is taken up by parsing the entire
code base, it is possible that the parsing in the SACA detection is slower than iClones
parsing, not the clone detection. Tree-sitter focuses on being performant on incremental
parsing, and therefore might not be as performant on the initial parse, which could
explain why the initial detection for both the CCDetect-LSP detections are slower than
iClones\footnote{Tree-sitter is not widely discussed in the literature, but is
open-source. This claim is based on the source code and discussions surrounding it.
(Github issues)}. In the case of our incremental algorithm, it is not unexpected that the
initial detection is slower than the SACA initial detection, and therefore also the
slowest algorithm overall for the initial detection. Because our incremental algorithm
does the same work as the initial detection, but also needs to build the wavelet-matrix
and dynamic extended suffix array data structures, it seems to consistently be the slowest
algorithm in the initial detection. This is one of the downsides of our incremental
algorithm, as the initial detection can be very slow for larger code bases, taking 3-4
minutes for the intellij-community code base.

For the subsequent incremental detections (revisions 1-9), we see that depending on the
size of the code base, and the size of the edits, our incremental detection and iClones
are generally much faster than the SACA detection, as expected. For the smallest code
bases (WorldWind and neo4j), our incremental detection seems to be the fastest for smaller
edits (INS10, DEL10), while iClones seems to be faster for the larger edits (INS100,
DEL100). It seems that when the code bases increase in size, our incremental detection
scales better than iClones and the SACA detection. The incremental detection is able to
process updates to intellij-community (5.8MLOC) in less than 1 second for smaller edits
(INS10, DEL10) and less than 2 seconds for the larger edits (INS100, DEL100).
Elasticsearch (3.2MLOC) was the largest code base that iClones could be run on with 16GB
RAM, and for this code base it seems that our incremental detection starts to outperform
the other two detection algorithms in all cases. It would be interesting to see if this
trend continues for even larger code bases and continue to compare CCDetect-LSP
incremental detection with iClones, but this would require more RAM to benchmark.

In terms of stability of performance, it seems that the SACA detection is naturally very
stable between each revision, as it is recomputing the entire suffix array anyway. The two
incremental algorithms are more unstable in terms of running time, and this stems from the
fact that some incremental updates are less computationally heavy than others. We cannot
speak to the implementation of iClones, however, we often see that iClones takes an extra
revision or two before it stabilizes at a lower running time. We can also sometimes see
spikes in performance for iClones (such as in figure \ref{fig:neo4j}). Our incremental
algorithm has a running time which fluctuates mostly by how much time updating the suffix
array takes. The other phases are generally either very stable (source-mapping) or are too
fast to differentiate between revisions (parsing, fingerprinting). The time to update the
suffix array is highly dependent on how many suffixes needs to be reordered. The number of
suffixes being reordered directly affects how many LCP values need to be updated, which is
often the bottleneck of the suffix array update. In our testing we examined two different
code bases of size ${\sim}2.2\text{MLOC}$, graal and flink. We hypothesized that graal
would have slower updates, as its $\text{LCP}_\text{avg}$ was more than double that of
flink. According to Salson et al.~\cite{DynamicExtendedSuffixArraysReorderings}, this
should on average lead to more suffixes being reordered. However, our results show that
the running time for updates in these code bases are very similar. The reason why we do
not see much of a difference could be because the $\text{LCP}_\text{avg}$ differences in
our code bases is too small to see any noticeable effect by doubling the
$\text{LCP}_\text{avg}$. Inspecting the numbers by Salson et
al.~\cite{DynamicExtendedSuffixArraysReorderings}, shows that the number of suffixes
needing to be reordered for lower $\text{LCP}_\text{avg}$ values were often below $20$ and
that the number of reordered suffixes only drastically increases for
$\text{LCP}_\text{avg}$ in the hundreds. These values are more realistic for inputs such
as DNA sequences than for our source code fingerprints. It therefore seems that source
code is a good candidate to apply the dynamic suffix array update algorithm to, as updates
in source code will on average mean few suffixes need to be reordered.

With these results, it is clear that CCDetect-LSP incremental detection is fast enough to
be used in an IDE environment for many code bases at least up to ${\sim}2.2$MLOC (flink,
graal), as most updates can be processed in under $1$ second for these code bases. For
code bases of larger sizes (elasticsearch, intellij), small edits (INS10, DEL10) can still
be processed in under $1$ second, but larger edits (INS100, DEL100) can start to take some
time, as the processing time exceeds $1$ second and can be closer to $2$ seconds. If the
programmer is generally only performing small edits in few files, and not doing
large-scale refactoring across many files at once, CCDetect-LSP is likely fast enough to
feel ``real-time'' even for these large code bases, but might feel sluggish to update on
such larger refactoring operations.

\section{Memory usage}

In \cref{evaluation}, we also saw the results of the memory usage evaluation for the
different tools. This section will discuss these results and consider which factors
increase the memory usage of each tool, and what this means for CCDetect-LSP when used in
an IDE environment.

The results show a clear picture on how the memory usage scales with the number of lines
of code. The SACA detection has the lowest peak memory usage, which is expected, since it
only needs to store the extended suffix array in array form, which is more memory
efficient than the dynamic structures of the incremental algorithms. The main memory usage
for the SACA detection is the extended suffix array, which is not much larger than the
source-mapping information.

For the incremental algorithms it is clear that CCDetect-LSP incremental detection has a
lower memory usage than iClones, but the memory usage is still quite high compared to the
SACA detection. Inspecting the JProfiler memory overview shows that the main memory
bottleneck of CCDetect-LSP incremental detection is the wavelet-matrix, and the dynamic
extended suffix array data structures. These two data structures take up about the same
amount of memory, and since they both are pointer-based structures representing values for
the entire fingerprint, these two take up a lot of memory combined. For iClones, we see
that the memory usage is even more severe, seemingly doubling the memory usage of
CCDetect-LSP incremental detection. JProfiler reports that it is the suffix tree data
structure which is the memory bottleneck in iClones. 

While our incremental detection manages to lower its memory usage compared to iClones, the
memory usage is still quite high for an IDE tool. For a larger code base such as
elasticsearch and intellij-community, one would likely require a computer with at least
16GB RAM in order to justify running CCDetect-LSP incremental detection in the IDE. Again,
it seems like CCDetect-LSP incremental detection is a good fit for code bases up to
${\sim}2.2MLOC$ (graal, flink), where the memory usage is lower than 4GB.

\section{Clones detected}

In \cref{evaluation} we saw that CCDetect-LSP identifies ${\sim}99.98\%$ of type-1 clones in
the BigCloneBench dataset. The results showed that CCDetect-LSP was able to identify
practically all type-1 clones, which gives us confidence that our algorithm is correct in
terms of the reported clones. BigCloneBench also reported that CCDetect-LSP was able to
detect some type-2 clones, but as previously stated, this is accidental.

We decided to store and represent clones as clone classes, rather than clone pairs. We
chose this because it gives a clearer picture to the programmer of all the clones of a
certain code snippet, rather than having to navigate multiple times between multiple
clones to determine all the clones of the snippet. Baars et al. also claims that storing
clone classes is advantageous in a refactoring scenario, which is intuitive, as
refactoring a set of clones likely leads to a better result than refactoring only a single
clone pair~\cite{TowardsAutomatedRefactoring}.

Detecting type-1 clones is likely the most important clones to detect in an IDE scenario,
but type-2 clones are also likely useful. Recall that type-2 clones are clones which are
structurally identical, but allows differences in literals, identifiers and types. Recall
also that both type-1 and type-2 clones are clones which are often good targets for
refactoring, since they can be parameterized to account for the differences in literals
and types, before they are merged into a single function/method. In the case of
CCDetect-LSP, we could normalize the input to allow detection of type-2 clones by
consistently fingerprinting tokens to the same value if they are allowed to be different.
Because CCDetect-LSP is language agnostic and relies only on a Tree-sitter grammar for a
given programming language, performing this normalization is more challenging than in
typical clone detectors. This is because CCDetect-LSP needs to know which type of tokens
should be consistently fingerprinted as the same value. One might also want to only detect
clones with consistently substituted parameters, where Baker's technique could be
implemented~\cite{Bakerdup}. This could be done by allowing the user to configure a list
of AST node types which should be consistently fingerprinted, but this also requires the
user to be intimately familiar with the node types in the AST of their chosen programming
language.

For type-3 clones we need to consider how useful it is to report these clones in an IDE
scenario. Recall that type-3 clones are clones which in addition to type-1 and type-2
clones, allow some leniency in terms of the code clones structure. Type-3 clones allows
tokens to be added, modified or removed and two type-3 clones are considered equal based
on some similarity threshold. Seeing such clones in the IDE could possibly lead to a lot
of noise, and CCDetect-LSP could possibly detect a lot of clones which are not necessarily
simple to refactor and remove. It would be interesting to see how many type-3 clones are
reported in large code bases and how much value it would provide to the user to list them
in the IDE. It would also be interesting to see how implementing a type-3 clone algorithm
such as Baker's algorithm~\cite{BakerSparseDynamicProgramming} would affect the speed of
the incremental detection, and if any type-3 detection algorithms can be made more
efficient for an incremental detection approach.

CCDetect-LSP identifies clones on a token level, unlike other tools which identifies
clones on a line level\cite{Zibran_real_time_search}. It is more fine-grained to detect
clones on a token level, and it could detect clones which would not be detected on a line
level without needing to format the source code. However, it is possible that we could see
a substantial performance and memory usage gain by instead fingerprinting entire lines
instead of each token. This would drastically reduce the size of the fingerprint, which
would make the suffix array updates faster and reduce the size of the dynamic extended
suffix array and wavelet matrix. However, as there are a lot more unique lines in a code
base than tokens, this would also drastically increase the alphabet size, $\theta$, which
will at least slow down operations on the wavelet matrix. This is a trade-off which would
be interesting to explore further.

\section{IDE and language agnostic clone detection}

CCDetect-LSP was implemented as a standalone program, which communicates with an IDE
client via LSP. This means that the tool is IDE agnostic in the sense that any IDE which
implements LSP can communicate with CCDetect-LSP and receive code clone information. With
our implementation we are able to list clones in the form of diagnostics, and allows the
programmer to navigate between all the matches of a clone via code-actions and/or
information embedded in the diagnostic. This shows that LSP is capable of supporting the
bare-minimum needed functionality of a clone detection tool. 

One downside to reporting code clones as diagnostics is that code clones are listed
together with all other diagnostics reported by other LSP servers. If the project contains
hundreds of clones this could make the diagnostics list hard to use for any other usage
than to list clones. A nice addition to the LSP protocol in this regard would be the
possibility of categorizing diagnostics, and letting the client filter or show only
diagnostics of a certain category. Another possibility is to let LSP clients know which
LSP server has sent which diagnostics, so that the LSP client can filter or show only
diagnostics of a certain LSP server at a time. For CCDetect-LSP it would be nice to be
able to have a separate view of only code clone diagnostics, which can be opened or closed
as needed, and allow another LSP server to be the ``main'' source of diagnostics. Ano

Another interesting feature which could be implemented with LSP is the concept of linked
editing. Linked editing means being able to edit two or more locations at the same time,
which could be used as an automatic way to apply an edit to all clones at once. LSP has
since version 3.16 had such a feature, but we are not sure how fitting this feature is for
code clone linked editing.

The algorithm of CCDetect-LSP is also language agnostic. We have shown that as long as
there exists a Tree-sitter grammar for a language\footnote{Tree-sitter can parse most
mainstream languages, see: \url{https://tree-sitter.github.io/tree-sitter/}}, it can be
parsed and fingerprinted. When the source code has been fingerprinted, there is no longer
any difference between detection for different languages. Therefore, CCDetect-LSP should
work for any language, and in our testing, there is nothing which hinders CCDetect-LSP
from working for any language we have tested, other than grammars not working correctly.
It is possible to forego parsing the source code, and only use a tokenizer to get the
tokens. From there one could use a language agnostic tokenization algorithm to avoid
needing a grammar at all, and makes the algorithm truly language agnostic. However, this
would not allow any sort of fragment selection, which likely leads to a lot of unnecessary
code being analyzed for clones, which both slows down the detection process, but can also
lead to a lot of noise in the clone list. For example in Java, a sorted list of imports at
the beginning of the file could easily be cloned in multiple files, without any real
possibility of being refactored to avoid this duplication. For this reason, we believe a
``grammar-pluggable'' approach with a parser generator such as Tree-sitter is a better
approach than a completely language agnostic one for the IDE scenario.
