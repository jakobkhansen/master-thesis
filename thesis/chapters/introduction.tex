\chapter{Introduction}

Refactoring is the process of restructuring source code in order to improve the internal behavior
of the code, without changing the external behavior~\cite[9]{fowlerrefactoring}.
Refactoring on source code is often performed in order to eliminate instances of bad
design quality in code, otherwise known as code smells.

A study conducted by Diego Cedrim et al. has shown that while programmers tend to refactor
smelly code, they are rarely successful at eliminating the smells they are
targeting~\cite{Rohit_Gheyi_Impact}. They also discovered that a large portion of
refactorings tend to make the code quality worse. Automated tools which help programmers
make better refactorings and perform code analysis could be a solution to this problem.

Duplicated code is a code smell which occurs in practically every large software project.
Code clone analysis (duplicate code analysis) has recently become a highly active field
of research and many tools have been developed to detect duplicated
code~\cite[7]{Inoue_introduction_to_cc}. 

\section{Motivation and problem statement}

Many tools and algorithms exist for duplicate code detection, however few of these have
the capability of efficiently detecting duplicated code in a real-time IDE environment.
Incremental algorithms which do not recompute all clones from scratch are interesting for
use-cases such as IDEs and different Git revisions of the same source code, but this type of
algorithm has not been thoroughly explored for code clone analysis.

The current landscape of clone detection algorithms and tools is therefore lacking in
terms of integration and support for fast incremental updates within an IDE. This limits
the ability for programmers to easily detect duplicated code as they work, as these tools
are also often limited to specific programming languages or IDEs.

Our proposed solution addresses this issue by introducing a new algorithm/pipeline that is
capable of detecting and updating code clones in real-time as source code changes, faster
than redoing the analysis from scratch. The tool which implements this algorithm is also
programming language and IDE agnostic, which allows programmers to seamlessly incorporate
the detection of duplicated code into their existing development environment.

\section{Structure}

The current chapter describes the motivation and problem statement of the thesis. Chapter
2 continues by giving some background on code clone analysis, some existing
tools and some preliminary algorithms and data structures used in the implementation.
Chapter 3 describes the implementation of the tool and the algorithm used for
incremental clone detection. In chapter 4, the tool is evaluated based on different
criteria, and compared to other existing solutions. Chapter 5 discusses the results of the
evaluation and discusses some choices made in the implementation. Chapter 6
concludes the thesis and lists future work.
