\chapter{Introduction}

Refactoring is the process of restructuring source code in order to improve the internal
behavior of the code, without changing the external behavior~\cite[9]{fowlerrefactoring}.
Refactoring source code is often performed in order to eliminate instances of bad code
quality, otherwise known as code smells.

A study conducted by Diego Cedrim et al. has shown that while programmers tend to refactor
smelly code, they are rarely successful at eliminating the smells they are
targeting~\cite{Rohit_Gheyi_Impact}. Most refactorings performed were either
``smell-neutral'', meaning that the targeted code smell is not eliminated, or ``stinky'',
meaning that the refactoring introduced more code smells than they eliminated. Automated
tools that help programmers make better refactorings and perform code analysis could be a
solution to this problem. 

Duplicate code, often called code clones, is code which is more or less copied to
different locations in a code base. Code clones occur in practically every large software
project, and code clone analysis has therefore become a highly active field of research in
the last decade. Many tools and algorithms have been developed to detect, manage and
refactor code clones~\cite[6]{Inoue_introduction_to_cc}. Code clone detection in large
software can be a time-consuming process, a consequence of this is that few existing tools
are highly integrated into the workflow of software developers. This thesis will therefore
focus on efficient code clone detection and describes a novel algorithm and tool for
detecting code clones in a real-time programming environment.

\section{Motivation and problem statement}

Many tools and algorithms exist for code clone detection. However, few of these have the
capability of efficiently detecting clones in a real-time programming environment. A
problem with existing algorithms is that most of them need to rerun the entire analysis
whenever a change to a file happens. Incremental algorithms that do not recompute all
clones from scratch are therefore interesting for use-cases such as in IDEs while
programming and for analyzing different revisions of the same source code, but this type
of algorithm has not been thoroughly explored for code clone analysis.

Our proposed solution and tool addresses this issue by introducing a new algorithm that is
capable of detecting and updating existing code clones as source code changes, which aims
to be faster than redoing the analysis from scratch. CCDetect-LSP, the tool which
implements this algorithm is also programming language and IDE agnostic, which allows
programmers to seamlessly incorporate the detection of duplicated code into their existing
development environment.

\section{Our contribution}

CCDetect-LSP provides code clone detection capabilities in a real-time IDE environment.
The tool allows the user to list and interact with all the clones in the code base, jump
between matching clones, and get quick feedback while editing code in order to determine
which clones are introduced/eliminated.

Existing incremental clone detection tools either do not fit into an IDE scenario, are
limited in terms of what clones they display, or have not been shown to scale well in
terms of processing time or memory usage for larger codebases. Therefore, this thesis will
focus on the following areas.

\textbf{Incremental clone detection:} The main focus of this thesis is making the tool
efficient in terms of incrementally updating the analysis whenever edits are performed in
the IDE. Most clone detection algorithms either only list clones of a specific code
snippet, or calculates clones from scratch in a manner which is too slow for an IDE
scenario. Our algorithm is based on a novel application and extension of dynamic suffix
arrays for clone detection, which can be efficiently updated and find clones. While suffix
arrays have been used for clone detection before~\cite{SHINOBI}, we are not aware of any
other attempts to use suffix arrays in an incremental setting. With this algorithm,
CCDetect-LSP can display all clones in the entire code base at once, and efficiently
update the displayed clones whenever a file is edited.

\textbf{IDE and language agnostic clone detection:} CCDetect-LSP gives programmers the
ability to view clones in their IDE. Utilizing features of the language server protocol
(LSP) such as diagnostics and code-actions, the tool provides clone analysis to any editor
which implements the LSP protocol. As far as we are aware there are no other clone
detection tools which utilizes LSP to provide clone analysis. In addition, the tool is
also language agnostic in the sense that it only needs a grammar for a language to analyze
it. 

\section{Structure}

Chapter 2 provides some background on code clone analysis, some existing tools and some
preliminary algorithms and data structures used in the implementation. Chapter 3-5
describes the implementation of CCDetect-LSP and the algorithms used for initial and
incremental clone detection. In chapter 6, the tool is evaluated based on multiple
criteria, and compared to other existing solutions. Chapter 7 discusses the results of the
evaluation. Chapter 8 concludes the thesis and lists future and related work.
