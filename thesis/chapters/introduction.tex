\chapter{Introduction}

Duplicate code, often called code clones in the literature, is code which is more or less
copied to different locations in a code base. Code clones occur in practically every large
software project, and code clone analysis has therefore become a highly active field of
research in the last decade. While code clones are not always harmful to a code
base~\cite[209]{Inoue_introduction_to_cc}, code clones are generally considered to
negatively affect maintainability, analyzability and extensibility of
software~\cite[22]{Inoue_introduction_to_cc}.

Code clone detection in large code bases can be a time-consuming process. A consequence of
this is that few existing tools are highly integrated into the workflow of software
developers and are not designed to be run on a code base every time the code base changes.
This thesis will therefore focus on efficient code clone detection and presents a novel
algorithm and tool for detecting code clones in a real-time programming environment. In
addition, the thesis explores the topics of both programming language agnostic clone
detection, and IDE agnostic tooling, which are also features of our tool.

\section{Motivation and problem statement}

Many tools and algorithms have been developed to detect, manage and refactor code
clones~\cite[6]{Inoue_introduction_to_cc}. However, few of these have the capability of
efficiently detecting clones in a real-time programming environment. A problem with
existing algorithms is that most of them need to rerun the entire analysis whenever a
change to a file in the code base occurs. Incremental algorithms that do not recompute all
clones from scratch are therefore interesting for use-cases such as while programming in
an integrated development environment (IDE) and for analyzing different revisions of the
same source code, but this type of algorithm has not been thoroughly explored for code
clone analysis. Existing incremental algorithms either do not scale well in terms of time
or memory, or are not designed to be run in an IDE environment.

Our proposed solution and tool addresses this issue by introducing a new algorithm that is
capable of detecting and updating existing code clones as source code changes, which aims
to be faster than rerunning the analysis from scratch. CCDetect-LSP\footnote{CCDetect-LSP source code is
available here: \url{https://www.github.com/jakobkhansen/CCDetect-lsp}}$^{,}$\footnote{Source
code DOI release: \url{https://doi.org/10.5281/zenodo.7933569}}, the tool which
implements this algorithm, is also programming language- and IDE agnostic, allowing
programmers to seamlessly incorporate the detection of duplicated code into their existing
development environment.

\section{Our contribution}

CCDetect-LSP provides type-1 and configurable type-2 code clone detection capabilities in
a real-time IDE environment. The tool allows the user to list and interact with all the
clones in the code base, jump between matching clones, and get quick feedback while
editing code in order to determine if clones are introduced or eliminated.

Existing incremental clone detection tools either do not fit into an IDE scenario, are
limited in terms of what clones they display, or have not been shown to scale well in
terms of processing time or memory usage for larger codebases. Therefore, this thesis will
focus on the following areas of code clone analysis.

\textbf{Incremental clone detection:} The main focus of this thesis is making the tool
efficient in terms of incrementally updating the analysis whenever edits are performed in
the IDE. Most clone detection algorithms either only list clones of a specific code
snippet, or computes clones from scratch in a manner which is too slow for an IDE
scenario. Our algorithm is based on a novel application and extension of dynamic extended
suffix arrays for clone detection, which can find clones and be efficiently updated
whenever the code base changes. While suffix arrays have been used for clone detection
before~\cite{SHINOBI}, we are not aware of any other attempts to use suffix arrays in an
incremental setting. CCDetect-LSP can display all clones in the entire code base at once,
and efficiently update the list of clones whenever a file is edited. Our results
demonstrate that this algorithm scales better than a non-incremental suffix array
algorithm and an existing incremental algorithm in terms of time, when small edits are
applied to a code base.

\textbf{IDE and language agnostic clone detection:} CCDetect-LSP gives programmers the
ability to view clones in their IDE. Utilizing features of the Language Server Protocol
(LSP) such as diagnostics and code-actions, the tool provides clone analysis to any editor
which implements LSP~\cite{lsp}. As far as we are aware there are no other clone detection
tools which utilizes LSP to provide clone analysis to IDEs. In addition, the tool is also
language agnostic in the sense that it only needs a grammar for the parser generator
Tree-sitter, for it to be analyzed~\cite{treesitter}. 

\section{Structure}

\Cref{background} provides background on code clone analysis, existing tools and
preliminary algorithms and data structures used in the implementation. Chapter
\ref{lspimplementation}-\ref{dynamicdetection} describes the implementation of
CCDetect-LSP and the algorithms used for initial and incremental clone detection. In
\Cref{evaluation}, the tool is evaluated based on multiple criteria, and compared to other
existing solutions. \Cref{discussion} discusses the results of the evaluation and lists related
work. \Cref{conclusion} lists future work and concludes the thesis.
