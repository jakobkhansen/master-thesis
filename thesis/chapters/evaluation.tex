\chapter{Evaluation}
\label{evaluation}

In this chapter, CCDetect-LSP will be evaluated based on different criteria, which combined
will provide a basis for evaluating the tool as a whole.

Since the tool is focused on efficient detection of code clones, real-time performance of
the tool will be a high priority in its evaluation. We will compare the time of the
initial detection with the incremental updates. Note that we will also distinguish between
the initial detection where parsing the entire code base is necessary, and subsequent
detections which still constructs the suffix array from scratch, but does not require
parsing the entire code base. We will call this type of detection the SACA detection,
while the detection which uses the dynamic extended suffix arrays will be called the
incremental detection. This is to show that the dynamic extended suffix array is faster
for this use case than building the suffix array from scratch, even when the initial
parsing is removed from the picture. Performance will be evaluated in two ways:

\begin{itemize}
    \item Complexity analysis of phases in initial detection and incremental detection
    \item Performance comparison of initial detection, incremental detection and another
        tool on code bases of different sizes
\end{itemize}

In addition, we will use the BigCloneBench~\cite{BigCloneBench} and
BigCloneEval~\cite{BigCloneEval} to verify that CCDetect-LSP correctly identifies type-1
clones.

\section{Time complexity of detection}

In this section we will conduct an informal runtime analysis of each phase of the initial,
SACA and incremental detection to argue that the complexity of the incremental detection
is more efficient than the initial detection on average. In each phase we will argue the
run time in terms of Big O notation.

The initial detection runs in $O(n)$ time where $n$ is the number of characters in the
code base. The bottleneck of the initial detection is reading and parsing all the content
in each file. Tree-sitter generates Generalized LR (GLR) parsers~\cite{GLR}, which in the
worst-case have a $O(n^3)$ running time, but $O(n)$ for any deterministic grammar. As most
programming languages have deterministic grammars, we will assume that the running time of
parsing with Tree-sitter takes $O(n)$ time. After the initial parsing, the SACA detection
runs in $O(f)$ time, where $f$ is the size of the fingerprint and $f \ll n$. The runtime
is $O(f)$ because the suffix array construction is performed for every update, which takes
linear time in the size of the input, which is the fingerprint. The extraction of clones
from the LCP array also runs in $O(f)$ as it is a single scan over the LCP array. The
final source-mapping is a bit more complicated, taking $O(\vert\text{clones}\vert \times
\log (\vert\text{documents}\vert))$, this is because for each clone, we binary search the
list of documents to find the correct document for that clone. This is highly likely to be
less time consuming than the suffix array construction, as the number of documents and
number of clones are usually multiple orders of magnitude lower than the size of the whole
code base. Therefore, we get a final runtime of $O(f) + O(\vert\text{clones}\vert \times
\log(\vert\text{document}\vert))$, where $O(f)$ is highly likely to be the factor which
grows faster.

For the incremental detection, we have already parsed the code base and built the index
and dynamic extended suffix array structure for the code base. Afterwards, when an edit
$E$ is performed in a document $D$, extracting the edit operations takes $O(\vert D\vert +
\vert E\vert^2)$ where $\vert E\vert \leq \vert D\vert$. Note that the size of the edit is
calculated as the area which $E$ covers, meaning that if an edit consists of changing a
token at the beginning of the file, and a token at the end of the file, $\vert E\vert
\approx \vert D\vert$. Also note that $\vert D\vert$ is the size of the fingerprint for
the document $D$. We get this time complexity because Hirschberg's algorithm runs in $O(n
\times m)$ where in our case, $n \approx m$. If the size of the edit is contained in a
smaller area, we apply the optimization which reduces the size of the edit by comparing
characters at the beginning and end of the string, as discussed in
\cref{dynamicdetection}. This processes takes $O(\vert D\vert)$, and afterwards
Hirschberg's algorithm takes in worst-case $O(\vert E\vert^2)$, depending on how small the
previous optimization made the input strings.

The complexity of dynamically updating the extended suffix array is actually slower than a
linear time SACA algorithm in the worst-case. The worst-case scenario when
inserting/deleting a character in the fingerprint is that every single suffix needs to be
reordered, meaning we have $O(f)$ reorderings, where each reordering takes $O(\log(f))$
time, as it requires deleting and inserting an element in the dynamic extended suffix
array. This is worse than the $O(f)$ running time of the SACA algorithm, however in
practice, the algorithm is highly likely to be faster. In practice, Salson et
al.\cite{DynamicExtendedSuffixArraysReorderings} have shown that on average, the number of
reorderings required for an insertion/deletion in a suffix array is highly correlated with
the average LCP value of the input. Their data shows that for multiple different types of
data such as genome sequences and english text, the average LCP value of the input is
generally magnitudes lower than the input size. In our experience, this applies to source
code as well, as the average LCP values in the fingerprint of code bases we have tested on
have all had an average LCP values well below $100$. This is intuitive, as lower LCP
values mean that the insertion/deletion will affect the ordering of fewer suffixes in the
input. With this information, it would be more accurate to downplay the importance of the
$O(f)$ number of reorderings in our analysis, and we therefore claim that the average
running time of inserting/deleting a character is closer to $O(\log(f))$. We extend this
to account for multiple insertions/deletions as well, so for each edit operations which
was computed in the last phase, we perform an insertion/deletion. Therefore, the total
running time of this phase on average is closer to $O(\vert\text{edits}\vert \times
\log(f))$.

\Todo{Need to show average LCP value in performance benchmark to build on this claim.}

\Todo{Complexity of clone extraction and source-mapping in dynamic structures}

Finally, we have the clone extraction and source-mapping phases. Recall that in the
dynamic detection clone extraction phase, we had stored all nodes with an LCP value above
the token threshold, and iterate over those to determine which of them are clones or not.
In the worst-case every index in the LCP array would be above the token threshold, which
would be an $O(f)$ time traversal. However, this is highly unlikely, and instead the
running time should be closer to $O(\vert\text{clones}\vert)$. Since the nodes with LCP
value above the token threshold are either clones, or contained clones.

\Todo{Statistic showing how many nodes above the token threshold}

\section{Performance comparison}

\Todo{Flame graphs?}

\section{Comparison with iClones}

\section{Memory usage}

\section{Verifying clones with BigCloneBench}
