% Created 2017-11-22 Wed 11:51
% Intended LaTeX compiler: pdflatex
\documentclass[11pt]{article}
\usepackage[utf8]{inputenc}
\usepackage[T1]{fontenc}
\usepackage{graphicx}
\usepackage{grffile}
\usepackage{longtable}
\usepackage{wrapfig}
\usepackage{rotating}
\usepackage[normalem]{ulem}
\usepackage{amsmath}
\usepackage{textcomp}
\usepackage{amssymb}
\usepackage{capt-of}
\usepackage{hyperref}
%\usepackage{minted}
\usepackage[margin=15mm,top=05mm]{geometry}


\usepackage{charter}       %% font

\usepackage{etex}
\usepackage{hyperref}



\usepackage[dvipsnames]{xcolor}
\newcommand{\myred}[1]                  {\textcolor{BrickRed}{#1}}

\usepackage{amsmath,amssymb}
\usepackage{subfigure,url}
\usepackage{array,longtable}
%\usepackage[a4paper,width=1.2\textwidth,height=1.3\textheight,twosideshift=0pt]{geometry}
%\usepackage[margin=25mm, top=33mm, bottom=28mm]{geometry}
\usepackage{pstricks,pst-node,psfrag}
\usepackage{epsfig}
\usepackage{hyperref}
%\usepackage{color}
\usepackage{supertabular}
\usepackage{colortbl}[2001/02/13]                %% colored tables, shaded boxes etc
                                                 %% be careful, don't use too old
\definecolor{lightpurple}{RGB}{197,180,227}
                                                 %% versions here!
\def\grayscale{0.8}                              %% defaul gray
\newcommand{\gc}{\cellcolor{lightpurple}}   %% gray cell                   
%\usepackage{times} 

%%%%%%%%%%%%%%%%%%%%%%%%%%%%%%%%%%%%%%%%%%%%%%%%%%%%%%%%
%w
\usepackage[sl]{titlesec}        %% sl = easy setting of styl

%\titleformat{\section}[block]{\bfseries\Large}{
%  %\filleft
%  \MakeUppercase{\scriptsize Abschnitt} \large \thesection\;\hrulefill}{1.8ex}{%
%  \filright\Huge}%[\vspace{1ex}\titlerule]

\titleformat{\part}[block]{\bfseries\Huge\sf}{}{0pt}{}[]
%\titleformat{\section}[block]{\bfseries\large\sf}{}{0pt}{}[\titlerule]
\titleformat{\section}[block]{\bfseries\Large}{}{0pt}{}[\titlerule]
\titleformat{\subsection}[block]{\bfseries\large}{}{0pt}{}[]
\titleformat{\subsubsection}[runin]{\bfseries\normalsize}{}{0pt}{}[]




\newlength{\tabone}
\newlength{\tabtwo}
\newlength{\tabthree}
\newlength{\tabfour}

\setlength{\tabone}{6cm}
\setlength{\tabtwo}{9cm}
\setlength{\tabthree}{4.8cm}
\setlength{\tabfour}{1cm}


%%%


%%% Local Variables: 
%%% mode: latex
%%% TeX-master: "projectdescription"
%%% End: 

\usepackage[backend=biber,firstinits=true,doi=false,isbn=false,url=false]{biblatex}
\addbibresource{refs.bib}
\date{December 2021}
\title{\myred{Refactoring}\vspace{-1cm}}
\begin{document}



\maketitle
\thispagestyle{empty}

\begin{center}
\begin{tabular}{ll}
\gc candidate & Jakob Konrad Hansen\\
\gc supervisors & Volker Stolz and Lars Tveito\\
\gc group & SIRIUS\\
\gc type & 60ECTS\\
\gc study program & Informatics: Programming and System Architecture\\
\gc planned date of completion & May 2023\\
\end{tabular}
\end{center}

\section*{Short description}
\label{sec:org4ecb37b}

The task is to implement and research refactoring for the ABS programming language
\textsubscript{\cite{DBLP:conf/fmco/JohnsenHSSS10}}.

\section*{Background and motivation}

Refactoring is the process of changing a software system in such a way that it does not
alter the external behavior of the code yet improves its internal structure. In essence
when you refactor you are improving the design of the code after it has been
written.\textsubscript{\cite{Fowler1999}}


The thesis will be based around implementing refactoring support and researching
refactorings for the ABS programming language. We want to offer users of the language more
refactorings and a satisfactory method to apply them. To do this we will build upon an
existing Xtext grammar\textsubscript{\cite{xtext}} and work with the LSP
protocol\textsubscript{\cite{lsp}}. A proposal to extend the LSP protocol might be made in
order to accomplish this goal. With this as a basis, the temporary thesis statement and
research questions will be

\begin{itemize}
    \item How should refactoring be implemented for ABS?
    \item Is the resulting approach generalizable to other programming languages?
    \item Which refactorings can we safely implement?
\end{itemize}

\section*{Methodology}

The thesis will consist of both theoretical and practical work. The theoretical work will
consist of reading relevant existing research and conceptualizing refactorings. The
practical work will consist of implementing software which will provide these
refactorings. That will be a process of specifying requirements for the software,
implementing the software and then evaluating if the software meets the requirements.

\printbibliography
\end{document}

