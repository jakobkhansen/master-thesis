\documentclass[12pt]{article}

\setlength\parindent{0pt}

\usepackage[english]{babel}
\usepackage{graphicx}
\usepackage{amsmath}
\usepackage{tikz}
\usepackage{hyperref}
\usepackage{graphicx}
\usepackage[a4paper, total={}]{geometry}
\setlength{\headsep}{5pt}
\setlength{\parskip}{1em}
\usepackage{listings}

\definecolor{pblue}{rgb}{0.2,0.2,0.7}
% \definecolor{pgreen}{rgb}{0,0.5,0}
\definecolor{pred}{rgb}{0.7,0.2,0.2}
\definecolor{pgrey}{rgb}{0.46,0.45,0.48}

\lstset{
  backgroundcolor = \color{lightgray},
  language=Bash,
  showstringspaces=false,
  columns=flexible,
  basicstyle={\small\ttfamily},
  numbers=none,
  numberstyle=\tiny\color{gray},
  keywordstyle=\color{blue},
  commentstyle=\color{dkgreen},
  stringstyle=\color{mauve},
  breaklines=true,
  breakatwhitespace=true,
  tabsize=3
}

\hbadness=10000

\usetikzlibrary{automata, positioning, arrows}

\tikzset{
    ->,
    node distance=3cm,
    initial text=$ $,
}

\title{Real-time management of code clones in an IDE environment}

\author{Jakob Hansen}

\date{\today}

\begin{document}
\maketitle

\tableofcontents

\section{Introduction}

Refactoring is the process of restructuring code in order to improve the internal behavior
of the code, without changing the external behavior.\cite{fowlerrefactoring}

Code clones are a problem ...

Many tools exist which ...

\section{Background}

\subsection{Software quality}

\subsubsection{Bad smells}

\subsubsection{Software quality metrics}

\subsubsection{How refactoring affects software quality}

\subsubsection{Duplicated code}

Write about what duplicated code is, how it affects software and some statistics on
duplicated code (need reference)

\subsection{Code clones}

\subsubsection{Code clone types}

Code clones are generally classified into four types.\cite{Inoue_introduction_to_cc} These
types classify code snippets as code clones with an increasing amount of leniency.
Therefore Type-1 code clones are very similar, while Type-4 clones are not necessarily
similar at all. However, all code clones do still have the same functionality, it is the
syntactic and structual differences which distinguish the types. The set of code clones
classified by a code clone type is also a subset of the next type, meaning all type-1
clones are also type-2 clones, but not vice versa.

The code clone types are defined as follows:

\textbf{Type-1} clones are syntactically identical. The only differences allowed are elements
without meaning, like comments and white-space. 

\textbf{Type-2} clones are structurally identical. Possible differences include
identifiers, literals and types. 

\textbf{Type-3} clones are structurally similar, but not equal. Differences include
statements which are added, removed or modified. For this clone type one needs to
determine a threshold $\theta$ which determines how structurally different snippets can be
to be considered Type-3\cite{Inoue_introduction_to_cc}.

\subsection{Code clone detection}

\subsection{Code clone management}

\section{The way forward}

\bibliography{refs}
\bibliographystyle{plain}

\end{document}
